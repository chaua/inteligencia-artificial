% =================================================================
% =                                                               =     
% =================================================================
% -----------------------------------------------------------------
% - Author:     Chaua Queirolo                                    -
% -----------------------------------------------------------------
\documentclass[a4paper,11pt]{karticle}    

% =================================================================
% PACKAGES
% =================================================================

% Configuration
\usepackage{dirtree}

% =================================================================
% HEADER
% =================================================================

\title{\tb{Prova 1: artigo}}
\author{Prof. Chauã Queirolo}
\date{} 

\disciplina{Inteligência Artificial}
\ano{2020/2}

% =================================================================
% DOCUMENT
% =================================================================
\begin{document}

\maketitle

\section*{Descrição}

Escreva um artigo em \LaTeX descrevendo a resolução, por meio de estratégias de busca, dos seguintes problemas: (1)~cubo de Rubik, (2)~missionários e canibais, (3)~problema das $n$ rainhas, e (4)~\ti{Sudoku}.

O artigo deverá analisar a natureza de cada um dos problemas, e quais algoritmos de busca são mais indicados para cada problema. Além disso, o artigo deverá descrever a modelagem do problema para sua resolução por meio de busca, informando: (1)~estado inicial, (2)~ações, (3)~modelo de transição, (4)~custo do caminho, e (5)~teste do objetivo. A descrição da modelagem pode fazer uso de gráficos e figuras ilustrativas.

O documento deve apresentar as seguintes artefatos:

\begin{enumerate}
    \item Título, nome e email institucional do autor.
    \item Introdução.
    \item Fundamentação teórica: explicação breve dos algoritmos de busca existentes e dos que podem ser utilizados.
    \item Análise dos problemas: descrever cada um dos problemas, quais algoritmos devem ou não ser utilizados, e a modelagem do problema para cada algoritmo.
    \item Considerações finais.
    \item Referência bibliográficas.
\end{enumerate}


\subsection*{Cubo de Rubik}

Cubo de Rubik, também conhecido como cubo mágico, é um quebra-cabeça tridimensional, inventado pelo húngaro Erno Rubik em 1974. Originalmente foi chamado o ``Cubo Mágico'' pelo seu inventor, mas o nome foi alterado pela Ideal Toys para ``cubo de Rubik''. O quebra-cabeças consiste em um cubo, no qual, cada uma das suas seis faces está dividida em nove partes, 3x3, num total de 26 peças que se articulam entre si devido ao mecanismo da peça interior central fixa, oculta dentro do cubo. 

Um algoritmo que conseguisse resolver qualquer cubo de Rubik no menor número de movimentos possíveis é designado por "algoritmo de Deus". Em 2005, o menor número de movimentos para resolver o cubo era de 28. Em 2007, passou a 26. Em 2010, foi provado que o número exato é 20. Para chegar a esse cálculo, alguns matemáticos, um engenheiro do Google e um programador dividiram o problema em 2.217.093.120 partes. 

\subsection*{Missionários e canibais}

Três canibais e três missionários estão viajando juntos e chegam à margem de um rio. Eles desejam atravessar para a outra margem para, desta forma, continuar a viagem. O único meio de transporte disponível é um barco que comporta no máximo duas pessoas. Há uma outra dificuldade: em nenhum momento o número de canibais pode ser superior ao número de missionários pois desta forma os missionários estariam em grande perigo de vida. 

\subsection*{Problema das $n$ rainhas}

O problema das oito rainhas é o problema matemático de dispor oito rainhas em um tabuleiro de xadrez de dimensão 8x8, de forma que nenhuma delas seja atacada por outra. Para tanto, é necessário que duas damas quaisquer não estejam numa mesma linha, coluna, ou diagonal. Este é um caso específico do Problema das $n$ rainhas, no qual temos $n$ rainhas e um tabuleiro com $n \times n$ casas(para qualquer $n \ge 4$).

\subsection*{\ti{Sudoku}}

O \ti{Sudoku} é um quebra-cabeça baseado na colocação lógica de números. O objetivo do jogo é a colocação de números de 1 a 9 em cada uma das células vazias numa grade de 9x9, constituída por 3x3 subgrades chamadas regiões. O quebra-cabeça contém algumas pistas iniciais, que são números inseridos em algumas células, de maneira a permitir uma indução ou dedução dos números em células que estejam vazias. Cada coluna, linha e região só pode ter um número de cada um dos 1 a 9. 



% -----------------------------------------------------------------
\begin{info}{Informações Gerais}
\begin{itemize}
  \item \tb{Data de entrega}: 07/10
  \item \tb{Número de integrantes}: individual
\end{itemize}
\end{info}

O artigo deverá estar formatado como \ti{article}, fonte tamanho 11pt e duas colunas. O documento deverá ter no máximo 8 páginas. O arquivo pdf deverá estar disponível no repositório em: 

\begin{itemize}
  \item \tb{inteligencia-artificial/prova01/artigo.pdf}
\end{itemize}


% -----------------------------------------------------------------
\section*{Entrega}

\begin{itemize}
    \item Os trabalhos devem ser entregues via \tb{github}\footnote{\url{https://github.com/}}.
    \item Criar um repositório chamado \tb{inteligencia-artificial}.
    \item Adicionar o usuário do professor (\tb{chaua}) aos colaboradores do projeto.
    \item Criar o arquivo README.md com o curso, disciplina, nome do aluno.
    \item Submeter no \tb{AVA} o artigo em pdf. 
\end{itemize}


\end{document}

